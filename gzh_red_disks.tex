
% CHAPTER:  1
% (Note: cannot have a footnote on a word within the \chapter{} construct, it does not work)
\chapter{GZH red disk fraction}
\label{chap:gzh_red_disks}

It is well known that most galaxies tend to exist in one of two populations: blue, late-type disks exhibiting active star formation, and red, early-type ellipticals showing little to no signs of recent star formation \citep{Strateva2001,Baldry2004}. The division between the two color populations is quite distinct when visually represented on a color-magnitude (CMD) or color-color diagram. As shown in the CMD in Figure~\ref{fig:CMD}, galaxies tend to populate in one of two regions: the ``red sequence'' in the upper band, which contains predominently early-type galaxies, and the ``blue cloud'' in the lower, containing mostly late-type spirals. This bimodality in the color-morphology relationship of galaxies has been so widely accepted that color is often used as a proxy for morphological classification in large samples of galaxies (e.g. \citet{Cooray2005,Lee2007,Salimbeni2008,Simon2009}), where expert visual classification is not feasable on such scales (see also: Chapter~\ref{chap:methodology}), while color measurements are more easily available. 

\begin{figure}
\centering
\includegraphics[width=0.75\textwidth]{figures/cmd.png}
\caption{cmd figure}
\label{fig:CMD}
\end{figure}


The relatively tight correlation suggests an evolutionary link between a galaxy's dynamical history (traced by its morphology) and stellar content (traced by its color). In the simplest interpretation, it could be deduced that galaxies tend to begin their lives as young, star-forming disks, until some mechanism (secular or external) causes star-formation to cease while the galaxy simultaneously undergoes a morphological tranformation from disk to spheroidal. 

The advent of larger surveys and more reliable methods for measuring morphology (independently of color) has allowed for more nuanced interpretations of the simple model. For instance, the degree of incompleteness in the color-morphology relationship is now much more realized, with the recent identifcations of significantly large samples of red spirals and blue ellipticals. Using morphological classifications from GZ1, \citet{Masters2010} found 6\% of a sample of $\sim$5000 spirals to be red; similarly, \citet{Schawinski2009} found 6\% of early-type galaxies to be blue. The existence of these objects may represent transition phases in the pathway from the blue cloud to the red sequence, and also give insight into what processes may quench or initiate star-formation without inducing a morphological change, or visa versa.

Another probe for understanding the transition from blue cloud to red sequence is the ``green valley'', the intermediate region between the two. In addition to the ellipticals in the blue cloud and spirals in the red sequence, galaxies of all types residing in the green valley were thought to represent the transition stages of this evolutionary pathway. The intermediate colors in this region indicate a recent quenching of star-formation \citep{Martin2007,Salim2007}, and the dearth of galaxies here (as compared to the high populations in the red sequence and blue cloud) initially suggested that the quenching process initiating transition across the CMD is very rapid.

A closer look at the populations within the green valley show that the processes causing galaxies to evolve from the blue cloud to red sequence may be very different. \citet{Schawinski2009} studied the morphological distribution (measured by the GZ1 project) of $\sim$4000 green-valley galaxies, finding that late-type and early-types likely go through two different evolutionary tracks. For late-types, the quenching process is gradual, and initiated by a cutoff of a gas reservoir. Galaxies quenched recently in this way would populate the green valley at $z=0$, and those which quenched at an earlier time would be currently identified as red passive disks. Whether these red disks continue to evolve into spheroidals via some process after the initial quenching is unclear from a local Universe analysis. For early-types, the quenching is rapid and probably external and violent, thus triggering the morphological change from disk to spheroidal.  

Analysis of the color-morphology relationship in the local Universe has revealed a close but imperfect bimodality as well as proposed mechanisms by which galaxies undergo different quenching processes, driving their evolution along the CMD. Even more may be revealed by studying the different populations as a function of cosmic time, which is becoming more possible with the data from large high-redshift surveys such as COSMOS and deep imaging via HST-ACS. It has been established now, for instance, that the bimodality does exist out to $z\sim1$ \citep{Bell2004,Cirasuolo2007,Mignoli2009} and possibly beyond \citep{Giallongo2005,VanDokkum2006,Franzetti2007,Cassata2008}. What requires further study is how exactly the proportions change at different epochs.



\subsection{Quenching Mechanisms}
\label{ssec:qmechs}
An isolated galaxy will eventually cease to form new stars as it naturally exhausts its limited supply of gas. The time-scale for complete consumption can be estimated from the amount of gas in a typical galaxy and the rate at which it is consumed through star formation: $\tau \sim M_{gas}/\dot{M}_{gas}$ \citep{Larson1980}, and is expected to range from 1-3 Gyr. Most galaxies do not exist in such isolation; the exchange of matter in the galaxy due to its surroundings can disrupt and often accelerate the depletion of a galaxy's gas reservoir. Quenching is defined as any process which drives the shutting-down of star formation in this way. This section will introduce different proposed quenching mechanisms, some of which are internal (driven by the galaxy's structure or components), or external (driven by direct influence of the surrounding environment). 

\subsubsection{Ram Pressure Stripping}

As a galaxy moves through the intracluster medium (ICM), it experiences ram pressure $P_{ram} = \rho_e v^2$, where $\rho_e$ is the density of the ICM and $v$ is the velocity of the galaxy \citep{Gunn1972}. The force per area required to hold gas onto the traveling galaxy is $F/A = 2\pi G \sigma_{s} \sigma_{g}$, where $\sigma_{s}$ and $\sigma_{g}$ are the star and gas surface densities, respecitively. If a galaxy is moving fast enough, or the ICM density is large enough, the ram pressure can exceed this force and consequently rip the gas from the galaxy; this process is known as ram pressure stripping. Evidence of this effect is seen observationally in asymmetries of the disk in spirals (a common example is NGC 4402, which has a bowed appearance and a one-sided concentration of dust, believed to be the effects of the galaxy struggling to hold onto gas on the outer regions of the disk) and truncated radial density profiles. Simulations \citep{Steinhauser2016} show that extreme cases of ram pressure stripping can completely strip a galaxy of its cold gas, causing a rapid quenching on timescales of a few hundred Myr. More mild cases, on the other hand, can actually temporarily increase star formation, which quickly uses up the available cold gas, and eventually quenches the galaxy on timescales similar to natural isolation, $\sim$ 1 Gyr. \citet{Fillingham2016} find a mass dependence on the efficiency of this process: they find RPS to be very efficient and rapid for galaxies $M_{*} = 10^{8-9}M_{\odot}$ for a range of halo host properties, suggesting RPS may be the dominant quenching mechanism for low-mass galaxies.  

\subsubsection{Strangulation and Harassment}

Even if the ram pressure exerted by the ICM is not strong enough to completely remove all of the gas from a galaxy, it may be just strong enough to strip the outer hot gas which would have otherwise cooled and replenished the cold gas reservoir. This process is appropriately defined as strangulation, where star formation ceases after the inital cold gas is used up \citep{Larson1980}. More frequent and violent encounters can increase star-formation in a similar process known as harassment \citep{Moore1996}. These can lead to a compression of the cold gas causing a temporary and intense burst of star formation, depleting it completely on a time scale of $\sim$1-2Gyr \citep{Kawata2007}. \citet{Moore1999} show through simulations that harassment can be powerful enough to alter the morphology of low-mass, low-surface brightness galaxies. 

\subsubsection{AGN feedback}
AGN are believed to play a strong role in the regulation of star-formation in their host galaxies via AGN \emph{feedback}, whereby accretion onto the central SMBH generates strong outflows of energetic material and hard radiation; these AGN-driven winds may then terminate star-formation by heating the gas or expelling it completely from the galaxy. This effect was first proposed as an important quenching mechanism through the development of theoretical models aiming to reproduce the observed local-Universe luminosity function. The bright end, where there is a sharp break in the observed number density of highly luminous galaxies, tends to only be reproducable in models which incorporate AGN feedback to suppress star-formation as galaxies build up their mass \citep{Benson2003, DiMatteo2005, Bower2006, Croton2006, Somerville2008}. One of the leading observational arguments for this effect is the high fraction of AGN in the green valley \citep{Martin2007a,Schawinski2010}, suggesting that AGN may be responsible in part for transitioning galaxies from the blue cloud to the red sequence. \citet{Smethurst2016} found strong evidence for rapid and recent quenching through an analysis of star formation histories of a large population of AGN hosts, indicating that AGN-feedback can play a strong role in the quenching process.  

\subsubsection{Mergers}
Mergers are a well-known driver of quenching \citep{Peng2010}, and are perhaps the dominant mechanism in particular for central galaxies \citet{Smethurst2017} whose dense environments give an increased probability of galaxy-galaxy interaction. Simulations demonstrate that these events cause a mechanical disruption of the merging disks \citep{Pontzen2017} which triggers powerful, brief starbursts \citep{Barnes1996,Hopkins2006}, yielding star formation rates up to twice that observed for their isolated counterparts \citep{Mihos1994}, resulting in a quench after the gas required for star formation is rapidly depleted. Major (1:1 mass ratio) events have been shown to strongly disrupt the morphology of the interacting galaxies. It is hypothesized modern-day ellipticals were primarily formed from the mergers of disk galaxies \citep{Toomre1977,Schweizer1982,Schweizer1990}, which has been supported both by simulations \citep{Mihos1996,Pontzen2017} and observations \citep{Schweizer1982,Wright1990,Stanford1991}, whereby recently-quenched galaxies were shown to follow $r^{1/4}$ light profiles. 


\section{Sample Selection}
\label{sec:reddisksample}

The parent sample of galaxies in this paper is drawn from the Galaxy Zoo: Hubble (GZH) catalog \citep{Willett2016}, which provides morphological classifications for galaxies sourced from the HST Legacy Surveys. From the main catalog we select galaxies with imaging from the Cosmic Evolution Survey (COSMOS, \citet{Scoville2007}) in the redshift range $0<z<1$. From this, we apply a magnitude cut of $M_{r^{+}}<-20.5$ to create a volume-limited sample. Rest frame NUV-r and r-J colors are taken from the UltraVISTA catalog \citep{McCracken2012,Ilbert2013}.

A sample of non-clumpy disk galaxies was identified using the morphological classifications provided by GZH. The sample includes subjects which meet the following criteria: $\rm f_{features} > 0.30$ and $\rm f_{clumpy,no} > 0.30$, where $\rm f$ is the debiased vote fraction. It was required that at least 20 votes were given for each question ($\rm N_{smooth~or~features} \ge 20$ and $\rm N_{clumpy} \ge 20$) to reduce uncertainty in the vote fractions.

\section{Galaxy Classification: Identifying the Passive Population}
\label{sec:colorcolor}
To classify the galaxies as quiescent or star-forming, a method similar to that described by \citet{Ilbert2013} (hereafter I13) was used, which implements a rest-frame NUV-$r^{+}$ versus $r^{+}$-J diagnostic. Here are some reasons these colors are great (NUV-r:) \citep{Arnouts2007a,Salim2005a,Wyder2007},\citep{Martin2007}

The demarcation line to separate the quiescent and active populations at $z=1$ is adopted from I13, which defines the quiescent galaxies as those which satisfy: $M_{NUV}-M_{r^{+}} > 3(M_{r^{+}}-M_{J})+1$ and $M_{NUV}-M_{r^{+}} > 3.1$. I13 applies this criteria to all galaxies in a range of $0.2<z<3$, although it performs best at separating the two populations in the redshift bin $0.7<z<1.2$, where $>98\%$ of galaxies identified as quiescent exhibited star formation rates less than $log(SFR) = -11$ (see Figure 3 of I13). Therefore this work uses the I13 separation criteria at $z=1$, and computes the evolution of the demarcation lines as a function of redshift to $z=0$. 

The evolution of $r-J$ and $NUV-r$ colors was measured using a stellar population synthesis model from \citet{Bruzual2003}. An instantanious-burst model (ssp) was chosen from the Padova1994 track to represent the color evolution of a passively evolving galaxy, with a metallicity $Z=0.008=.4Z_{\sun}$, which is the typical metallicity of passive galaxies with mass $9 < log(M_{*}/M_{\odot}) < 10$ (\citet{Peng2015}, Figure 2a), chosen to correspond to the median mass of the sample ($log(M_{*}/M_{\odot})=9.7)$. Figure~\ref{fig:bcmodel} shows the evolution of the two colors as a function of redshift, where the single starburst at $t=0$ was placed at $z=6$. A linear relationship was fit to the data within the range $0<z<2$, and the slope was used to redefine the demarcation lines in five redshift bins: one with central value $z=0.007$ (used to classify the SDSS ferengi2 sample), and four with central values $z$ = [0.30,0.50,0.70,0.90] with widths $\Delta z=0.2$. The quiescent galaxies are thus defined in these bins as those that satisfy:

\begin{equation}
M_{NUV}-M_{r^{+}} > 3.1 + a_{1}(z)
~\rm and~
M_{NUV}-M_{r^{+}} > 3(M_{r^{+}}-M_{J} + a_{2}(z))+ a_{1}(z) + 1  
\end{equation}

where $a_{1}(z) = [0.54,0.38,0.27,0.16,0.05]$ and $a_{2}(z) = [0.19,0.14,0.10,0.06,0.02]$. 

\begin{figure}
\centering
\includegraphics[width=\textwidth]{figures/hr_m52_evo.pdf} 
\caption{Evolution of colors using stellar population synthesis models. Galaxy was assumed to have formed at $z=6$ for plotting purposes.}
\label{fig:bcmodel}
\end{figure}

\section{Using Ferengi2 to correct for incompleteness in the red disk fraction}

described in previous section that it's difficult or impossible to disentangle whether a small vote fraction of \ffeatures corresponds to galaxies which are intrinsically smooth, or whose features have been washed out at high redshift. So while we can't correct the vote fraction for unique galaxies, we \emph{can} estimate the \emph{number} of disk galaxies we would fail to detect at a given redshift. 

To account for this incompleteness in disk detection, a correction factor $\xi$ is applied. This is defined as the number of disks detected divided by the true number of disks expected to exist in a given redshift interval: $\rm \xi(z)=N_{detected}/N_{true}$. Acknowledging that the completeness in disk detection may depend on galaxy color, the corrected red disk fraction can be calculated as:

\begin{equation}
f=\frac{N_{RD}\times \xi^{-1}_{red}}{N_{RD}\times \xi^{-1}_{red} + N_{BD} \times \xi^{-1}_{blue}}
\label{eqn:reddiskfraction}
\end{equation}

If there is no color bias in disk detection, $\xi_{red}=\xi_{blue}$, and this term cancels out, leaving the fraction unchanged. If there is a bias, however, the $\xi$ terms do not cancel, and the incompleteness in disk detection could have a large effect on the red disk fraction. Therefore a careful measurment of $\xi$ is estimated for both red and blue disk galaxies using the \ferengi2 set of simulated images.

\ferengi2 is the set of images created from 936 nearby ($z<0.01$) galaxies that were artificially redshifted to 8 redshifts between $z=0.3$ and $z=1$, giving a total of 7,488 simulated images (the sample is described in detail in Chapter~\ref{chap:ferengi}). The images were classified in Galaxy Zoo using the same decision tree as used for Galaxy Zoo Hubble. 134 highly inclined disk galaxies were removed from the sample by excluding any with $N_{edgeon}>20$ and $f_{not~edge-on}>=0.6$, using the vote fraction associated with the real galaxy image measured in GZ2. This cut was shown in Chapter~\ref{chap:baragn} to correlate well with inclination angle $cos(a/b)<67^\circ$. This was to exclude those which may be mis-classified due to dust-reddening (see section~\ref{sec:reddisksample}).  Using the NUV-J-R selection method described in section~\ref{sec:colorcolor}, the remaining sample was divided into a set of red sequence galaxies (259 per redshift bin) and blue cloud (543 per each redshift bin), shown in Figure~\ref{fig:ferengi2colorcolor}.
 

\begin{figure}
\centering
\includegraphics[width=.6\textwidth,trim={.5cm 0cm .5cm 0cm},clip]{figures/ferengi2_colorcolor.pdf}
\caption{Separation of the quiescent population (red sequence) and active population (blue cloud) of the \ferengi2 sample.}
\label{fig:ferengi2colorcolor}
\end{figure}


The completness values $\xi_{red}(z),\xi_{blue}(z)$ were then computed in varying bins of redshift for the red sequence and blue cloud galaxies separately. An example calculation of $\xi_{blue}$ in the $z=0.7$ bin is shown in Figure~\ref{fig:inc_subplot}. Each point represents a \ferengi2 galaxy, where the y-axis indicates the value of \ffeatures~measured in the image redshifted to $z=0.7$, and the x-axis indicates the value of \ffeatures~measured in the same galaxy redshifted to $z=0.3$. Disk galaxies are identified as those for which \ffeatures~$\ge0.3$. Since, on average, \ffeatures~decreases for the same galaxy as it is viewed at higher redshifts, the number of galaxies meeting this threshold is generally fewer at higher redshifts than lower redshifts. This is indicated by the dotted lines: galaxies to the right of the vertical dashed line at $\rm f_{features,z=0.3}=0.3$ are identified as disks at $z=0.3$; their sum is considered the ``true'' number of disks, $\rm N_{true}$. Similarly, the galaxies above the horizontal line at $\rm f_{features,z=0.7}=0.3$ are identified as disks at $z=0.7$; their sum is the ``detected'' number of disks at $z=0.7$, or $\rm N_{detected}$. As obvious in the figure, $\rm N_{detected}$ is in general much lower than $\rm N_{true}$, emphasizing the increasing difficulty in detecting features at higher redshifts. Their ratio is the completeness $\xi$; in this example $\xi_{blue}(z=0.7)=0.61$, meaning only 61\% of disks were detected at this redshift. 

\begin{figure}
\centering
\includegraphics[width=.65\textwidth]{figures/incompleteness_z7.pdf}
\caption{Example calculation of completeness $\xi$ at redshift $z=0.7$. Points represent \ferengi2 images classified in Galaxy Zoo. The y-axis corresponds to the value of \ffeatures~measured at the galaxy redshifted to $z=0.7$, and the x-axis corresponds to the value of \ffeatures~measured at the galaxy redshifted to $z=0.3$. On average, the \ffeatures~is lower at the higher redshift, indicating users on average have more difficulty identifying features in images at higher redshifts. The dotted lines correspond to \ffeatures=0.3, the threshold above which a galaxy is considered to have a disk. Galaxies to the right of the vertical dashed line were identified as disks at the lowest redshift $z=0.3$, the total number defined as $\rm N_{true}$, the true number of disks. Galaxies above the horizontal dash line were identified as disks at the higher redshift $z=0.7$, the total number defined as $\rm N_{detected}$. The ratio $\rm \xi=N_{detected}/N_{true}$ is the completeness value; in this example, only 61\% of disks were detected at $z=0.7$.}
\label{fig:inc_subplot}
\end{figure}

It was hypothesized that the completeness in disk detection may be a function of other parameters in addition to redshift. At fixed redshift, for example, it is reasonable to guess that features could be easier to detect galaxies that have higher mass, radius, or surface brightness. To test whether these parameters also impact the number of disks detected, the completeness was measured in fixed redshift bins as a function of surface brightness, effective radius, and mass. Surface brightness was measured using \sextractor{} calculations of {\tt MAG\_AUTO}, $b/a$ and $R_{e}$ measured in the \Iband{} band images, in the same way as described in Chapter~\ref{chap:ferengi}. The effective radius used was the 50\% {\tt FLUX\_RADIUS} converted in to kpc, and the masses used were the {\tt MEDIAN} values calculated in the MPA-JHU catalog.

Figure~\ref{fig:xi_v_sb} shows completeness as a function of redshift and surface brightness, for the red sequence and blue cloud galaxies. 8 redshift bins were further divided into bins of surface brightness with varying widths, where the sizes were chosen to satisfy that $N_detected + N_true \ge 10$ in each bin. This was chosen as a comprimise between having a sufficient number of galaxies in each bin to compute the completness fraction $\xi = N_{detected}/N_{true}$, and to have enough bins of surface brightness to measure a trend with confidence of completeness as a function of $\mu$. Visual inspection of the data did not suggest any relationship between the two. To be sure, the data were fit to a linear function in each redshift bin (Figure~\ref{fig:notlinear}). For each fit, a p-value representing a hypothesis test whose null hypothesis is that the slope is zero was computed. Only one reached the criteria $p<0.05$, but with a low $R^{2}$ value of 0.28 which is not considered large enough to represent a good fit. This process was repeated using effective radius and mass as parameters, with the same results. Therefore only redshift was used as a parameter which impacted completeness value with confidence.  


\begin{figure}
\centering
\includegraphics[width=\textwidth,trim={3cm 0cm 3cm 0cm},clip]{figures/xi_v_sb.pdf}
\caption{Completeness $\xi$ as a function of redshift and surface brightness for red sequence (left) and blue cloud galaxies (right).}
\label{fig:xi_v_sb}
\end{figure}

\begin{figure}
\centering
\includegraphics[width=\textwidth,trim={2cm 1cm 2cm 1cm},clip]{figures/notlinear.pdf}
\caption{Completeness $\xi$ as a function of redshift and surface brightness for red sequence (left) and blue cloud galaxies (right).}
\label{fig:notlinear}
\end{figure}


\begin{figure}
\centering
\includegraphics[width=.5\textwidth,trim={0cm 1cm 1cm 1cm},clip]{figures/completenessmoneyplot.pdf}
\caption{Completeness $\xi$ as a function of redshift moneyplot}
\label{fig:notlinear}
\end{figure}

\section{Results}
\label{sec:results}
In this section we present our results of the evolution of disc galaxies from $z=1$ to $z=0.2$ in a sample of 27,355 $COSMOS$ galaxies morphologically classified in GZH. We will show x, y, and z and talk about it. 

\subsection{The evolving passive disk fractions: $f_{R|D}$ and $f_{D|R}$} 

\begin{figure*}
\centering
\includegraphics[width=\textwidth,trim={0cm 0cm 2cm 1cm},clip]{figures/fractions_modeled.pdf}
\caption{\textbf{Left:} Passive disk fraction ($\rm N_{red~disks}/(N_{red~disks}+N_{blue~disks})$) vs redshift in four mass bins. \textbf{Right:} Fraction of disks on the red sequence ($\rm N_{red~disks}/(N_{red~disks}+N_{red~ellipticals})$) vs redshift in four mass bins. The solid lines represent the evolution of the fractions computed by the toy-model described in Section~\ref{sec:results} using rate parameters which yielded the lowest $\chi^2$.}
\label{fig:f_results}
\end{figure*}

\begin{figure}
\centering
\includegraphics[width=3.5in]{figures/cartoon.pdf}
\caption{Cartoon representing three evolutionary pathways represented in the toy-model described in Section~\ref{sec:results}. Path A represents an active star-forming galaxy which quenches without destroying the disk, becoming a red disk; the fraction of blue galaxies to follow this track is represented by $r_{BD \rightarrow RD}$ in the model. Path B represents a red disk morphologically transition to red ellipitcal; the fraction of red galaxies to follow this track is $r_{RD \rightarrow RE}$. Path C represents a blue disks simultaniously quenching and morphologically transforming to become a red elliptical; the fraction of blue galaxies to follow this track is $r_{BD \rightarrow RE}$. (The fourth driver of the relative numbers of different morphological types is $\kappa_{RE}$, the net merger rate of ellipticals in a given mass bin, and is not represented in the cartoon.) }
\label{fig:cartoon}
\end{figure}
The change in the relative number densities of active/passive disk/elliptical galaxies traces the dominant evolutionary pathways they follow at different mass thresholds. In Figure~\ref{fig:f_results} we measure these using the fractions defined in the previous section, $f_{R|D}$ (left panel) and $f_{D|R}$ (right panel) for four mass bins. We observe significantly different trends in $f_{R|D}$ for the two highest mass bins ($log(M/M_{\odot})>10.7$): for higher-mass galaxies, the fraction of red disks vs. all disks, within error, is either relatively flat or exhibits a small decrease, while the lower-mass galaxies have trends which increase sharply from $z=1$ to $z=0.3$. Similarly for the fraction of red disks on the red sequence: the highest mass bin $log(M/M_{\odot})>11.0$ decreases in $f_{D|R}$, while the lowest-mass bin increases sharply from $f_{D|R}\sim0.05$ to $\sim 0.2$.  

To gain a more intuitive understanding of how the trends of these fractions depend on the different possible evolutionary pathways and their transition rates, it is helpful to rewrite them in a reduced form: $f_{R|D} = ({1+\frac{N_{BD}}{N_{RD}}})^{-1}$ ; $f_{D|R} = ({1+\frac{N_{RE}}{N_{RD}}})^{-1}$. Using the former as an example: at zeroth order, it can be seen that if the number of red disks increases at a higher rate than an increase in blue disks in some time interval, the overall fraction $f_{R|D}$ would also increase. For a single mass bin, this scenario would be consistent with a model in which blue galaxies are quenching faster than they are entering the mass bin via star formation. A strong decrease in $f_{R|D}$ would, in constrast, indicate a higher rate of red disks exiting the mass bin than the blue disks; the strength of the decrease would depend on the relative frequencies of blue galaxies quenching to form new red disks and red disks merging to form red ellipticals. The precise relative values of these rates cannot be deduced by a simple by-eye analysis of the fraction, however.

We therefore implement a simple toy model to track the change in $f_{R|D}$ and $f_{D|R}$, given a range of parameters representing the quenching and morphological transformation rates for galaxies at fixed stellar mass. We begin by considering the rate of change in the number of blue disks ($dN_{BD}/dt$), red disks ($dN_{RD}/dt$), and red ellipticals ($dN_{RE}/dt$). In a given mass bin, the change in numbers for each population will depend on several parameters, illustrated visually in Figure~\ref{fig:cartoon}.

\subsubsection{Blue Disks}
First, galaxies in a blue bin may transition into a red disk bin via a quenching process that does not destroy its disk; we define this rate as $r_{BD \rightarrow RD}$, representing the fraction of blue galaxies to transition to red disks per Gyr (path A in Figure~\ref{fig:cartoon}). Blue galaxies may also exit a bin via a quenching process which \emph{does} destroy the disk; this fraction per Gyr we define as $r_{BD \rightarrow RE}$ (path C in Figure~\ref{fig:cartoon}).

The number of galaxies in a blue disk bin will also change due to star formation, which brings active galaxies from a lower mass bin into the current mass bin. To account for this term we use the formalism outlined by \citet{Peng2010}, in which this rate of change is given by $(\alpha + \beta)sSFR$. Here $\alpha = d\phi_{blue}/dm$ is the derivative of the mass function for blue galaxies, which equates to $\alpha = (1+\alpha_s) - m/M^*$ for a mass function described by the Schechter (1976) function. We use best-fit parameters for blue galaxies measured by Johnpaperetal, which give $\alpha_s = -1.4$ and $M^* = 10.28 ~(log(M/M_{\odot}))$. Following the method of \citet{Peng2010}, we let $\beta=0$, both for simplicity, and because their conclusions found not to be strongly dependent on $\beta$. Last, the specific star-formation rate is given by $sSFR(t) = 2.5(\frac{t}{3.5 Gyr})^{-2.2}Gyr^{-1}$ \citep{Peng2010}.

Accounting for all sources and sinks of blue disks entering or exiting a bin of given mass, the rate of change of blue disks can be written fully as:

\begin{equation}
\frac{dN_{BD}}{dt}\Big\rvert_{m} = \Big(-r_{BD \rightarrow RD} - r_{BD\rightarrow RE} -\alpha(m) sSFR(t) \Big)N_{BD}
\label{eqn:BD}
\end{equation}

\subsubsection{Red Disks}
Galaxies exiting a blue bin as they quenched without disrupting their disks enter the pool of red disks, increasing $N_{RD}$ for a given mass bin. Red disks also may undergo a morphological transformation, depleting the pool of red disks as they enter the red elliptical bin (path B in Figure~\ref{fig:cartoon}). The fraction of galaxies to undergo this pathway per Gyr is denoted as $r_{RD->RE}$. Combining these factors gives the expression: 

\begin{equation}
\frac{dN_{RD}}{dt}\Big\rvert_{m} = + r_{BD \rightarrow RD}N_{BD} - r_{RD \rightarrow RE}N_{RD}
\label{eqn:RD}
\end{equation}

\subsubsection{Red Ellipticals}
In this simple model, it is assumed that red, passive ellipticals are the final state in a typical galaxy's evolution. Therefore $N_{RE}$ will always be increasing from the transformation from blue disks and red disks to red ellipticals ($r_{BD \rightarrow RE}$, $r_{RD \rightarrow RE}$). However, the number of red ellipticals \emph{in a single mass bin} may still decrease due to ellipticals at the given mass merging to enter a bin of red ellipticals at a higher mass. Similarly, their number can increase as ellipticals from a lower mass bin merge to enter the current mass bin. A complete, semi-analytic model would consider this full range of possibilities and couple the resulting equations appropriately amongst all mass bins. For the purposes of this simple model, we opted to represent the total, net rate of change of the number of red ellipticals as a single parameter, $\kappa_{RE}$, which we note may be positive or negative, depending on whether more ellipticals are entering or leaving the given mass bin. 

\begin{equation}
\frac{dN_{RE}}{dt}\Big\rvert_{m} = \kappa_{RE} N_{RE}  
\label{eqn:RE}
\end{equation}

We exclude the contribution from blue ellipticals from the model because they represent only a small fraction of the total population (find references which quantify this.) We initialize our model using the observed relative numbers of blue disks, red disks, and red ellipticals measured at $z=1$, then use the model to compute their evolution to $z=0.3$ using a range of values for each of the four parameters in four mass bins. For $r_{BD \rightarrow RD}, r_{BD \rightarrow RE},$ and $r_{RD \rightarrow RE}$, we test 25 values between 0 and 1, and 25 values between -1 and 1 for $\kappa_{RE}$. We note that a complete model would explore time-varying rates, but for the purposes of simplicity in our toy-model we only experiment with static parameters. For each mass bin, the model was implemented for each permutation of the four rate parameters. The success of each run was evaluated using a $\chi^2$ metric; these results are shown for each mass bin in the corner-plot in Figure~\ref{fig:corner}. The bins are weighted by $1/\chi^2$, such that white regions represent the rate parameters that yield the lowest $\chi^2$, and black representing the largest.

\begin{figure*}
\centering
\includegraphics[width=\textwidth,trim={0cm 0cm 2cm 1cm},clip]{figures/corner_plot.pdf}
\caption{Results of the grid-search for the best-fit rate parameters $r_{BD \rightarrow RD}, r_{BD \rightarrow RE}, r_{RD \rightarrow RE}$, and $\kappa_{RE}$ for four mass bins. The units for all rate parameters is $\rm Gyr^{-1}$. 25 equally-spaced values were tested between (0,1) for each parameter, with the exception of $\kappa_{RE}$ which was tested for 25 values between (-1,1); these are represented by the 25 bins on each axis. Each bin is weighted by $1/\chi^2$, such that white regions correspond to parameters which produced the lowest $\chi^2$, and black representing the highest. There is a strong result in the dependence of $r_{BD \rightarrow RD}$ with mass, such that the fraction of blue disks which transition to red disks (ie, quench without disrupting the disk), increases for more massive galaxies. The other parameters are less constrained by this model; therefore a more complex semi-analytic model will be necessary for obtaining the precise values of these rates, and is the subject of future work.}
\label{fig:corner}
\end{figure*} 

We find a strong mass dependence on the fraction of blue galaxies to quench to red disks ($r_{BD \rightarrow RD}$), or Path A in Figure~\ref{fig:cartoon}. Our observations of $f_{R|D}$ and $f_{D|R}$ are most closely reproduced when $r_{BD \rightarrow RD}$ = [0.05, 0.07, 0.1, 0.2] $\rm Gyr^{-1}$ for masses $\rm log(M/M_{\odot})$ = [10.25,10.55,10.85,11.0]. [Note: need to measure peaks explicitely, current values are by-eye. Also calculate 1-sig errors.] These values for $r_{BD \rightarrow RD}$ correspond to the peaks of the 1-D histograms shown in Figure~\ref{fig:corner}. This increase of $r_{BD \rightarrow RD}$ with mass could suggest either: 1) more massive galaxies are more likely to undergo quenching processes which do not destroy their disks, or 2) less massive galaxies simply quench less frequently overall, via any pathway. 


 Analysis of the next parameter in the low mass bin, $r_{BD\rightarrow RE}$, suggests that the former is more likely, given the peak of $r_{BD \rightarrow RE}$ at $>0.9 ~\rm Gyr^{-1}$. The high rate of low-mass blue disks quenching to red ellipticals is evidence that they do not quench any less frequently than high mass galaxies, and the increase of $r_{BD \rightarrow RD}$ with mass is indeed consistent with quenching processes less likely to destroy the disk of massive galaxies. However, this result is not nearly as constrained, given the broad distribution of likelihoods for this parameter. $r_{BD \rightarrow RE}$ is even less constrained for all higher masses. The degeneracies evident in this rate and $r_{RD \rightarrow RE}$ make it clear that our model is not sufficient to constrain the relative fequencies of the processes involved in quenching and morphological transformations; a more sophisticated model with the adjustments we have described thus far would be necessary to paint the full picture. 



\section{Discussion}
\label{sec:Discussion}

We have explored the evolution of the passive disk population since $z=1$ by quantifying their abundances in terms of the fraction of disk galaxies that are red $f_{R|D}$ and the fraction of the red sequence occupied by disks $f_{D|R}$. For both fractions, we observed dependencies on both mass and redshift; this is a strong indication that passive disks play an important role in understanding the processes involved in the quenching and morphological transformation of galaxies. We identified three pathways to describe the evolution of a star-forming disk: $r_{BD \rightarrow RD}$ (the rate of blue disks quenching to red disks), $r_{BD \rightarrow RE}$ (the rate of blue disks quenching and transforming directly to red ellipticals), and $r_{RD \rightarrow RE}$ (the rate of red disks transforming to red ellipticals), and we argue that the relative frequencies of these rates drive the trends in $f_{R|D}$ and $f_{D|R}$. To begin quantifying the occurances of each of the pathways, we developed a toy model to reproduce $f_{R|D}$ and $f_{D|R}$ given some set of rates. Our model was able to constrain $r_{BD \rightarrow RD}$ to a reasonable degree of certainty, but the rest would require a more sophisticated model to deduce.

An analysis of $r_{BD \rightarrow RD}$ provides an estimate for the fraction of galaxies to go through a passive disk phase. For the most massive galaxies, we find $r_{BD \rightarrow RD}$ = 0.20 $\pm .1~Gyr^{-1}$, indicating that 20\% of massive galaxies become red disks at some point in their lifetime. Whether these 20\% tend to stay red disks or tend to evolve to ellipticals is unclear without a more solid constraint on $r_{RD \rightarrow RE}$. Using the same logic for the other three mass bins, we can find that [5\%, 7\%, 10\%] of galaxies with masses [10.25,10.55,10.85] become red disks in their lifetimes. These estimates are in agreement with \citet{Bundy2010} (hereafter B10) who estimate an upper limit of 60\% of massive galaxies to experience a red disk phase. Their estimation comes from a different approach to ours, via an analysis of the mass function of galaxies with different morphologies and the fractional contribution of disks on the red sequence $f_{D|R}$.  

Our estimation of the significance of the passive disk population is in agreement with B10 for the most massive galaxies, as is the downward trend in $f_{D|R}$ with redshift from $z=1$ to $z=0.3$. As suggested previously, a downward trend of $f_{D|R}$ represents either a depletion of the total pool of red disks (via a transformation to elliptical), but could also be an indication of an increase in the pool of red ellipticals (which could result from blue or red disks transforming morphology). In contrast, an upward trend is only possible via a pile-up of red disks, which is what we observe for the lower mass bins. Thus far our results are then consistent with a physical scenerio in which 1) more massive galaxies are more likely to become passive disks (given by the increase of $r_{BD \rightarrow RD}$ with increasing mass), and 2) less massive galaxies who enter a red disk phase are more likely than massive galaxies to stay in that phase, rather than transform to elliptical (given by the increase of $f_{D|R}$ from $z=1$ to $z=0.3$ for low mass bins).

Our first point is in agreement with literature which explores the unimodality of disk galaxies across a CMD \citep{Schawinski2014,Powell2017}. The smooth transition from the blue cloud to the red sequence in the distribution of low to medium mass disk galaxies is evidence for slow quenching timescales. For higher mass disk galaxies, the unimodality is broken, suggesting a more rapid quenching. This could be due to a higher merger rate for more massive galaxies, or as suggested by \citet{Schawinski2014}, evidence for a mass-quenching effect, in which the galaxy's halo reaches a critical mass whereby the gas is inhibited from cooling sufficiently to continue star-formation \citep{Kormendy2004,Dekel2006,Peng2010}. This result is also consistent with B10, who observe the strongest decrease in $f_{D|R}$ from $z=1$ to $z=0.3$ for their most massive galaxies.

Our second point is in disagreement with B10, who observe downward trends in $f_{D|R}$ in low mass galaxies. At the lowest redshift bin ($z\sim0.3$), we measure similar absolute fractions of disks occupying the red sequence for all masses. However, B10 find their contribution to increase at higher lookback time to $z=1$, while we find a decreasing contribution. The fact that our results agree for the highest mass at all redshifts, but only at the lowest redshift for lower masses, suggests the differences may be attributed in biases in morphological classification. B10 indentifies early and late-type disk galaxies using ZEST \citep{Scarlata2007} morphologies, which they acknowledge are biased towards disk classification for faint apparent magnitudes, which tend to be attributed to the lowest mass, highest redshift objects. This bias could influence their observed increase in red sequence disks toward $z=1$ for low masses. On the opposite end, GZ classifications tend to be biased towards elliptical morphologies at fainter magnitudes. We attempted to quantify and correct for this effect as described in Section~\ref{ssec:ferengi}, but if our correction was under-estimated, that may have driven the decreasing abundance of disk galaxies observed at increasing redshift for low masses. However, it has been shown in the local Universe that red disk galaxies tend to be more massive, as in \citet{Masters2011}. If this is true at all epochs, we would not expect such a significant contribution of red disks for low mass galaxies as found in B10. 
\section{Conclusions}
\label{sec:conclusions}

We have investigeted the influence of the passive disk population by measuring the relative abundances of blue disks, red disks, and red ellipticals since $z=1$ using morphological classifications from Galaxy Zoo: Hubble and rest-frame colors from UltraVISTA. Using data from artificially-redshifted \ferengi2 images to quantify the known redshift bias in the GZ classifications, we implemented a correction to the incompleteness in the number of disks detected as a function of redshift. The relative numbers were measured in terms of the fraction of disk galaxies that are red $f_{R|D}$ and the fraction of disk galaxies on the red sequence $f_{D|R}$. A simple toy-model was developed to simulate the evolution of these fractions as a function of the rates of three dominant evolutionary pathways: $r_{BD \rightarrow RD}$, the rate of blue disks quenching to red disks, $r_{BD \rightarrow RE}$,the rate of blue disks quenching and transforming directly to red ellipticals, and $r_{RD \rightarrow RE}$,the rate of red disks transforming to red ellipticals. Our main conclusions are as follows:

\begin{itemize}

\item{$f_{R|D}$ and $f_{D|R}$ decrease from $z=1$ to $z=0.3$ for massive galaxies, and increase for the least massive galaxies.}

\item{We estimate as high as 20\% (+-error) of massive ($log(M/M_{\odot})>11$) galaxies experience a passive disk phase. This fraction decreases with mass, down to 5+- \% for galaxies $log(M/M_{\odot})<10.4$.}

\item{Low mass galaxies which experience a passive disk phase are more likely than massive galaxies to remain disks, while massive galaxies are more likely to continue thier evolution by transforming to passive ellipticals. To quantify and validate this result would require a more sophisticated model than the simple approached used in this project.}


\end{itemize}



