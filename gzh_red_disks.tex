
% CHAPTER:  1
% (Note: cannot have a footnote on a word within the \chapter{} construct, it does not work)
\chapter{GZH red disk fraction}
\label{chap:gzh_red_disks}

It is well known that most galaxies tend to exist in one of two populations: blue, late-type disks exhibiting active star formation, and red, early-type ellipticals showing little to no signs of recent star formation. The division between the two color populations is quite distinct when visually represented on a color-magnitude (CMD) or color-color diagram. As shown in the CMD in Figure~\ref{fig:CMD}, galaxies tend to populate in one of two regions: the ``red sequence'' in the upper band, which contains predominently early-type galaxies, and the ``blue cloud'' in the lower, containing mostly late-type spirals. This bimodality in the color-morphology relationship of galaxies has been so widely accepted that color is often used as a proxy for morphological classification in large samples of galaxies (e.g. \citet{Cooray2005,Lee2007,Salimbeni2008,Simon2009}), where expert visual classification is not feasable on those scales (see also: Chapter~\ref{chap:methodology}), while color measurements are more easily available. 

\begin{figure}
\centering
\includegraphics[width=0.75\textwidth]{figures/cmd.png}
\caption{cmd figure}
\label{fig:CMD}
\end{figure}

-morphology traces dynamics while color traces stellar content, so relationship implies these may evolve together to an extent

-however with improved means of measuring the morphology of large samples of galaxies, the imperfect bimodiality has become more apparent
GZ - big sample of passive disks (Masters) and big sample of blue ellipticals (Schawniski) in local universe. Conselice 2006 and Magnoli 2009 measure incompleteness

-transition - high redshift
-not only is the bimodiality broken in the local universe, but the populations themselves evolve with time. find some sources which cite proporations as function of redshift. 
