
% CHAPTER:  1
% (Note: cannot have a footnote on a word within the \chapter{} construct, it does not work)
\chapter{GZH red disk fraction}
\label{chap:gzh_red_disks}

It is well known that most galaxies tend to exist in one of two populations: blue, late-type disks exhibiting active star formation, and red, early-type ellipticals showing little to no signs of recent star formation \citep{Strateva2001,Baldry2004}. The division between the two color populations is quite distinct when visually represented on a color-magnitude (CMD) or color-color diagram. As shown in the CMD in Figure~\ref{fig:CMD}, galaxies tend to populate in one of two regions: the ``red sequence'' in the upper band, which contains predominently early-type galaxies, and the ``blue cloud'' in the lower, containing mostly late-type spirals. This bimodality in the color-morphology relationship of galaxies has been so widely accepted that color is often used as a proxy for morphological classification in large samples of galaxies (e.g. \citet{Cooray2005,Lee2007,Salimbeni2008,Simon2009}), where expert visual classification is not feasable on such scales (see also: Chapter~\ref{chap:methodology}), while color measurements are more easily available. 

\begin{figure}
\centering
\includegraphics[width=0.75\textwidth]{figures/cmd.png}
\caption{cmd figure}
\label{fig:CMD}
\end{figure}


The relatively tight correlation suggests an evolutionary link between a galaxy's dynamical history (traced by its morphology) and stellar content (traced by its color). In the simplest interpretation, it could be deduced that galaxies tend to begin their lives as young, star-forming disks, until some mechanism (secular or external) causes star-formation to cease while the galaxy simultaneously undergoes a morphological tranformation from disk to spheroidal. 

The advent of larger surveys and more reliable methods for measuring morphology (independently of color) has allowed for more nuanced interpretations of the simple model. For instance, the degree of incompleteness in the color-morphology relationship is now much more realized, with the recent identifcations of significantly large samples of red spirals and blue ellipticals. Using morphological classifications from GZ1, \citet{Masters2010} found 6\% of a sample of $\sim$5000 spirals to be red; similarly, \citet{Schawinski2009} found 6\% of early-type galaxies to be blue. The existence of these objects may represent transition phases in the pathway from the blue cloud to the red sequence, and also give insight into what processes may quench or initiate star-formation without inducing a morphological change, or visa versa.

Another probe for understanding the transition from blue cloud to red sequence is the ``green valley'', the intermediate region between the two. In addition to the ellipticals in the blue cloud and spirals in the red sequence, galaxies of all types residing in the green valley were thought to represent the transition stages of this evolutionary pathway. The intermediate colors in this region indicate a recent quenching of star-formation \citep{Martin2007,Salim2007}, and the dearth of galaxies here (as compared to the high populations in the red sequence and blue cloud) initially suggested that the quenching process initiating transition across the CMD is very rapid.

A closer look at the populations within the green valley show that the processes causing galaxies to evolve from the blue cloud to red sequence may be very different. \citet{Schawinski2009} studied the morphological distribution (measured by the GZ1 project) of $\sim$4000 green-valley galaxies, finding that late-type and early-types likely go through two different evolutionary tracks. For late-types, the quenching process is gradual, and initiated by a cutoff of a gas reservoir. Galaxies quenched recently in this way would populate the green valley at $z=0$, and those which quenched at an earlier time would be currently identified as red passive disks. Whether these red disks continue to evolve into spheroidals via some process after the initial quenching is unclear from a local Universe analysis. For early-types, the quenching is rapid and probably external and violent, thus triggering the morphological change from disk to spheroidal.  

Analysis of the color-morphology relationship in the local Universe has revealed a close but imperfect bimodality as well as proposed mechanisms by which galaxies undergo different quenching processes, driving their evolution along the CMD. Even more may be revealed by studying the different populations as a function of cosmic time, which is becoming more possible with the data from large high-redshift surveys such as COSMOS and deep imaging via HST-ACS. It has been established now, for instance, that the bimodality does exist out to $z\sim1$ \citep{Bell2004,Cirasuolo2007,Mignoli2009} and possibly beyond \citep{Giallongo2005,VanDokkum2006,Franzetti2007,Cassata2008}. What requires further study is how exactly the proportions change at different epochs.



\subsection{Quenching Mechanisms}
\label{ssec:qmechs}
An isolated galaxy will eventually cease to form new stars as it naturally exhausts its limited supply of gas. The time-scale for complete consumption can be estimated from the amount of gas in a typical galaxy and the rate at which it is consumed through star formation: $\tau \sim M_{gas}/\dot{M}_{gas}$ \citep{Larson1980}, and is expected to range from 1-3 Gyr. Most galaxies do not exist in such isolation; the exchange of matter in the galaxy due to its surroundings can disrupt and often accelerate the depletion of a galaxy's gas reservoir. Quenching is defined as any process which drives the shutting-down of star formation in this way. This section will introduce different proposed quenching mechanisms, some of which are internal (driven by the galaxy's structure or components), or external (driven by direct influence of the surrounding environment). 

\subsubsection{Ram Pressure Stripping}

As a galaxy moves through the intracluster medium (ICM), it experiences ram pressure $P_{ram} = \rho_e v^2$, where $\rho_e$ is the density of the ICM and $v$ is the velocity of the galaxy \citep{Gunn1972}. The force per area required to hold gas onto the traveling galaxy is $F/A = 2\pi G \sigma_{s} \sigma_{g}$, where $\sigma_{s}$ and $\sigma_{g}$ are the star and gas surface densities, respecitively. If a galaxy is moving fast enough, or the ICM density is large enough, the ram pressure can exceed this force and consequently rip the gas from the galaxy; this process is known as ram pressure stripping. Evidence of this effect is seen observationally in asymmetries of the disk in spirals (a common example is NGC 4402, which has a bowed appearance and a one-sided concentration of dust, believed to be the effects of the galaxy struggling to hold onto gas on the outer regions of the disk) and truncated radial density profiles. Simulations \citep{Steinhauser2016} show that extreme cases of ram pressure stripping can completely strip a galaxy of its cold gas, causing a rapid quenching on timescales of a few hundred Myr. More mild cases, on the other hand, can actually temporarily increase star formation, which quickly uses up the available cold gas, and eventually quenches the galaxy on timescales similar to natural isolation, $\sim$ 1 Gyr. \citet{Fillingham2016} find a mass dependence on the efficiency of this process: they find RPS to be very efficient and rapid for galaxies $M_{*} = 10^{8-9}M_{\odot}$ for a range of halo host properties, suggesting RPS may be the dominant quenching mechanism for low-mass galaxies.  

\subsubsection{Strangulation and Harassment}

Even if the ram pressure exerted by the ICM is not strong enough to completely remove all of the gas from a galaxy, it may be just strong enough to strip the outer hot gas which would have otherwise cooled and replenished the cold gas reservoir. This process is appropriately defined as strangulation, where star formation ceases after the inital cold gas is used up \citep{Larson1980}. More frequent and violent encounters can increase star-formation in a similar process known as harassment \citep{Moore1996}. These can lead to a compression of the cold gas causing a temporary and intense burst of star formation, depleting it completely on a time scale of $\sim$1-2Gyr \citep{Kawata2007}. \citet{Moore1999} show through simulations that harassment can be powerful enough to alter the morphology of low-mass, low-surface brightness galaxies. 

\subsubsection{Mass quenching}
\subsubsection{AGN feedback}
AGN are believed to play a strong role in the regulation of star-formation in their host galaxies via AGN \emph{feedback}, whereby accretion onto the central SMBH generates strong outflows of energetic material and hard radiation; these AGN-driven winds may then terminate star-formation by heating the gas or expelling it completely from the galaxy. This effect was first proposed as an important quenching mechanism through the development of theoretical models aiming to reproduce the observed local-Universe luminosity function. The bright end, where there is a sharp break in the observed number density of highly luminous galaxies, tends to only be reproducable in models which incorporate AGN feedback to suppress star-formation as galaxies build up their mass \citep{Benson2003, DiMatteo2005, Bower2006, Croton2006, Somerville2008}. One of the leading observational arguments for this effect is the high fraction of AGN in the green valley \citep{Martin2007a,Schawinski2010}, suggesting that AGN may be responsible in part for transitioning galaxies from the blue cloud to the red sequence. \citet{Smethurst2016} found strong evidence for rapid and recent quenching through an analysis of star formation histories of a large population of AGN hosts, indicating that AGN-feedback can play a strong role in the quenching process.  

\subsubsection{Mergers}

\section{Sample Selection}
\label{sec:reddisksample}
info to include:
cross match with ultravista for rest-frame colors, volume limit, morphological cuts (pfeatures, pclumpy, pedgeon)

\section{Galaxy Classification: Identifying the Passive Population}
\label{sec:colorcolor}
To classify the galaxies as quiescent or star-forming, a method similar to that described by \citet{Ilbert2013} (hereafter I13) was used, which implements a rest-frame NUV-$r^{+}$ versus $r^{+}$-J diagnostic. Here are some reasons these colors are great (NUV-r:) \citep{Arnouts2007a,Salim2005a,Wyder2007},\citep{Martin2007}

The demarcation line to separate the quiescent and active populations at $z=1$ is adopted from I13, which defines the quiescent galaxies as those which satisfy: $M_{NUV}-M_{r^{+}} > 3(M_{r^{+}}-M_{J})+1$ and $M_{NUV}-M_{r^{+}} > 3.1$. I13 applies this criteria to all galaxies in a range of $0.2<z<3$, although it performs best at separating the two populations in the redshift bin $0.7<z<1.2$, where $>98\%$ of galaxies identified as quiescent exhibited star formation rates less than $log(SFR) = -11$ (see Figure 3 of I13). Therefore this work uses the I13 separation criteria at $z=1$, and computes the evolution of the demarcation lines as a function of redshift to $z=0$. 

The evolution of $r-J$ and $NUV-r$ colors was measured using a stellar population synthesis model from \citet{Bruzual2003}. An instantanious-burst model (ssp) was chosen from the Padova1994 track to represent the color evolution of a passively evolving galaxy, with a metallicity $Z=0.008=.4Z_{\sun}$, which is the typical metallicity of passive galaxies with mass $9 < log(M_{*}/M_{\odot}) < 10$ (\citet{Peng2015}, Figure 2a), chosen to correspond to the median mass of the sample ($log(M_{*}/M_{\odot})=9.7)$. Figure~\ref{fig:bcmodel} shows the evolution of the two colors as a function of redshift, where the single starburst at $t=0$ was placed at $z=6$. A linear relationship was fit to the data within the range $0<z<2$, and the slope was used to redefine the demarcation lines in five redshift bins: one with central value $z=0.007$ (used to classify the SDSS ferengi2 sample), and four with central values $z$ = [0.30,0.50,0.70,0.90] with widths $\Delta z=0.2$. The quiescent galaxies are thus defined in these bins as those that satisfy:

\begin{equation}
M_{NUV}-M_{r^{+}} > 3.1 + a_{1}(z)
~\rm and~
M_{NUV}-M_{r^{+}} > 3(M_{r^{+}}-M_{J} + a_{2}(z))+ a_{1}(z) + 1  
\end{equation}

where $a_{1}(z) = [0.54,0.38,0.27,0.16,0.05]$ and $a_{2}(z) = [0.19,0.14,0.10,0.06,0.02]$. 

\begin{figure}
\centering
\includegraphics[width=\textwidth]{figures/hr_m52_evo.pdf} 
\caption{Evolution of colors using stellar population synthesis models. Galaxy was assumed to have formed at $z=6$ for plotting purposes.}
\label{fig:bcmodel}
\end{figure}

\section{Using Ferengi2 to correct for incompleteness in the red disk fraction}

described in previous section that it's difficult or impossible to disentangle whether a small vote fraction of \ffeatures corresponds to galaxies which are intrinsically smooth, or whose features have been washed out at high redshift. So while we can't correct the vote fraction for unique galaxies, we \emph{can} estimate the \emph{number} of disk galaxies we would fail to detect at a given redshift. 

To account for this incompleteness in disk detection, a correction factor $\xi$ is applied. This is defined as the number of disks detected divided by the true number of disks expected to exist in a given redshift interval: $\rm \xi(z)=N_{detected}/N_{true}$. Acknowledging that the completeness in disk detection may depend on galaxy color, the corrected red disk fraction can be calculated as:

\begin{equation}
f=\frac{N_{RD}\times \xi^{-1}_{red}}{N_{RD}\times \xi^{-1}_{red} + N_{BD} \times \xi^{-1}_{blue}}
\label{eqn:reddiskfraction}
\end{equation}

If there is no color bias in disk detection, $\xi_{red}=\xi_{blue}$, and this term cancels out, leaving the fraction unchanged. If there is a bias, however, the $\xi$ terms do not cancel, and the incompleteness in disk detection could have a large effect on the red disk fraction. Therefore a careful measurment of $\xi$ is estimated for both red and blue disk galaxies using the \ferengi2 set of simulated images.

\ferengi2 is the set of images created from 936 nearby ($z<0.01$) galaxies that were artificially redshifted to 8 redshifts between $z=0.3$ and $z=1$, giving a total of 7,488 simulated images (the sample is described in detail in Chapter~\ref{chap:ferengi}). The images were classified in Galaxy Zoo using the same decision tree as used for Galaxy Zoo Hubble. 134 highly inclined disk galaxies were removed from the sample by excluding any with $N_{edgeon}>20$ and $f_{not~edge-on}>=0.6$, using the vote fraction associated with the real galaxy image measured in GZ2. This cut was shown in Chapter~\ref{chap:baragn} to correlate well with inclination angle $cos(a/b)<67^\circ$. This was to exclude those which may be mis-classified due to dust-reddening (see section~\ref{sec:reddisksample}).  Using the NUV-J-R selection method described in section~\ref{sec:colorcolor}, the remaining sample was divided into a set of red sequence galaxies (259 per redshift bin) and blue cloud (543 per each redshift bin), shown in Figure~\ref{fig:ferengi2colorcolor}.
 

\begin{figure}
\centering
\includegraphics[width=.6\textwidth,trim={.5cm 0cm .5cm 0cm},clip]{figures/ferengi2_colorcolor.pdf}
\caption{Separation of the quiescent population (red sequence) and active population (blue cloud) of the \ferengi2 sample.}
\label{fig:ferengi2colorcolor}
\end{figure}


The completness values $\xi_{red}(z),\xi_{blue}(z)$ were then computed in varying bins of redshift for the red sequence and blue cloud galaxies separately. An example calculation of $\xi_{blue}$ in the $z=0.7$ bin is shown in Figure~\ref{fig:inc_subplot}. Each point represents a \ferengi2 galaxy, where the y-axis indicates the value of \ffeatures~measured in the image redshifted to $z=0.7$, and the x-axis indicates the value of \ffeatures~measured in the same galaxy redshifted to $z=0.3$. Disk galaxies are identified as those for which \ffeatures~$\ge0.3$. Since, on average, \ffeatures~decreases for the same galaxy as it is viewed at higher redshifts, the number of galaxies meeting this threshold is generally fewer at higher redshifts than lower redshifts. This is indicated by the dotted lines: galaxies to the right of the vertical dashed line at $\rm f_{features,z=0.3}=0.3$ are identified as disks at $z=0.3$; their sum is considered the ``true'' number of disks, $\rm N_{true}$. Similarly, the galaxies above the horizontal line at $\rm f_{features,z=0.7}=0.3$ are identified as disks at $z=0.7$; their sum is the ``detected'' number of disks at $z=0.7$, or $\rm N_{detected}$. As obvious in the figure, $\rm N_{detected}$ is in general much lower than $\rm N_{true}$, emphasizing the increasing difficulty in detecting features at higher redshifts. Their ratio is the completeness $\xi$; in this example $\xi_{blue}(z=0.7)=0.61$, meaning only 61\% of disks were detected at this redshift. 

\begin{figure}
\centering
\includegraphics[width=.65\textwidth]{figures/incompleteness_z7.pdf}
\caption{Example calculation of completeness $\xi$ at redshift $z=0.7$. Points represent \ferengi2 images classified in Galaxy Zoo. The y-axis corresponds to the value of \ffeatures~measured at the galaxy redshifted to $z=0.7$, and the x-axis corresponds to the value of \ffeatures~measured at the galaxy redshifted to $z=0.3$. On average, the \ffeatures~is lower at the higher redshift, indicating users on average have more difficulty identifying features in images at higher redshifts. The dotted lines correspond to \ffeatures=0.3, the threshold above which a galaxy is considered to have a disk. Galaxies to the right of the vertical dashed line were identified as disks at the lowest redshift $z=0.3$, the total number defined as $\rm N_{true}$, the true number of disks. Galaxies above the horizontal dash line were identified as disks at the higher redshift $z=0.7$, the total number defined as $\rm N_{detected}$. The ratio $\rm \xi=N_{detected}/N_{true}$ is the completeness value; in this example, only 61\% of disks were detected at $z=0.7$.}
\label{fig:inc_subplot}
\end{figure}

It was hypothesized that the completeness in disk detection may be a function of other parameters in addition to redshift. At fixed redshift, for example, it is reasonable to guess that features could be easier to detect galaxies that have higher mass, radius, or surface brightness. To test whether these parameters also impact the number of disks detected, the completeness was measured in fixed redshift bins as a function of surface brightness, effective radius, and mass. Surface brightness was measured using \sextractor{} calculations of {\tt MAG\_AUTO}, $b/a$ and $R_{e}$ measured in the \Iband{} band images, in the same way as described in Chapter~\ref{chap:ferengi}. The effective radius used was the 50\% {\tt FLUX\_RADIUS} converted in to kpc, and the masses used were the {\tt MEDIAN} values calculated in the MPA-JHU catalog.

Figure~\ref{fig:xi_v_sb} shows completeness as a function of redshift and surface brightness, for the red sequence and blue cloud galaxies. 8 redshift bins were further divided into bins of surface brightness with varying widths, where the sizes were chosen to satisfy that $N_detected + N_true \ge 10$ in each bin. This was chosen as a comprimise between having a sufficient number of galaxies in each bin to compute the completness fraction $\xi = N_{detected}/N_{true}$, and to have enough bins of surface brightness to measure a trend with confidence of completeness as a function of $\mu$. Visual inspection of the data did not suggest any relationship between the two. To be sure, the data were fit to a linear function in each redshift bin (Figure~\ref{fig:notlinear}). For each fit, a p-value representing a hypothesis test whose null hypothesis is that the slope is zero was computed. Only one reached the criteria $p<0.05$, but with a low $R^{2}$ value of 0.28 which is not considered large enough to represent a good fit. This process was repeated using effective radius and mass as parameters, with the same results. Therefore only redshift was used as a parameter which impacted completeness value with confidence.  


\begin{figure}
\centering
\includegraphics[width=\textwidth,trim={3cm 0cm 3cm 0cm},clip]{figures/xi_v_sb.pdf}
\caption{Completeness $\xi$ as a function of redshift and surface brightness for red sequence (left) and blue cloud galaxies (right).}
\label{fig:xi_v_sb}
\end{figure}

\begin{figure}
\centering
\includegraphics[width=\textwidth,trim={2cm 1cm 2cm 1cm},clip]{figures/notlinear.pdf}
\caption{Completeness $\xi$ as a function of redshift and surface brightness for red sequence (left) and blue cloud galaxies (right).}
\label{fig:notlinear}
\end{figure}


\begin{figure}
\centering
\includegraphics[width=.75\textwidth,trim={0cm 1cm 1cm 1cm},clip]{figures/completenessmoneyplot.pdf}
\caption{Completeness $\xi$ as a function of redshift moneyplot}
\label{fig:notlinear}
\end{figure}


\section{Simple model}

\begin{flalign}
&\frac{dN_{BD}}{dt}\Big\rvert_{m} = \Big(-r_{BD_{m} \rightarrow RD_{m}} - r_{BD_{m} \rightarrow RE_{2m}}\Big)N_{BD_{m}} -\alpha(m) sSFR(t) \nonumber  \\
&\frac{dN_{RD}}{dt}\Big\rvert_{m} = + r_{BD_{m} \rightarrow RD_{m}}N_{BD_{m}} - r_{RD_{m} \rightarrow RE_{2m}}N_{RD_{m}} \nonumber \\
&\frac{dN_{RE}}{dt}\Big\rvert_{2m} = + r_{BD_{m} \rightarrow RE_{2m}}N_{BD_{m}} + r_{RD_{m} \rightarrow RE_{2m}}N_{RD_{m}} + \kappa_m N_{RE_{2m}} \nonumber  \\  
\label{eqn:rateeqs}
\end{flalign}

\begin{flalign}
&\alpha(m) = \frac{d~\rm log~\phi}{d~\textrm{log}~m} = (1+\alpha_s) - \frac{m}{M^{*}}\nonumber  \\
&sSFR(m,t) = 2.5\Big(\frac{t}{3.5~Gyr}\Big)^{-2.2}  \nonumber  \\  
\label{eqn:as}
\end{flalign}

using the Schechter function  $\phi(m) = \phi^*(m/M^*)^{1+\alpha_s}e^{-m/M^*}$ using parameters measured by \citet{Peng2010}



