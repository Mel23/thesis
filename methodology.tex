%%%%%%%%%%%%%%%%%%%%%%%%%%%%%%%%
% methodology.tex: Citizen Science chapter
%%%%%%%%%%%%%%%%%%%%%%%%%%%%%%%%
\chapter{Methodology}
\label{chap:methodology}

\section{A Brief History of Galaxy Zoo}

The increasing accessibility of the Internet in the last decade has allowed scientists to ``outsource'' tasks online using citizen science, with huge success. The project Seti@Home \footnote{http://setiathome.berkeley.edu/} \citep{Anderson2002}, launched in May 1999, was one of the first projects that revealed the massive number of people willing and excited to help contribute to science. Since launch, over 5 million participants donated idle time on their computers to assist SETI in analyzing radio telescope data to help in the search for extraterrestrial life. Citizen scientists were also extemely interested in taking in an even more active role in research, as seen in a later project Stardust@Home \footnote{http://stardustathome.ssl.berkeley.edu/}, in which volunteers searched for dust grains in data via a web interface. This project engaged over 20,000 volunteers, and those who discovered dust grains were invited to become co-authors on the announcement papers. Early citizen science projects such as these inspired the launch of Galaxy Zoo.

The real need for a faster method of obtaining galaxy morphologies for large samples was realized in 2007 by graduate student Kevin Schawinksi, who was studying populations of elliptical galaxies as work for his PhD thesis at Oxford University. At the time, the accepted and fastest method for identifying early-type galaxies (in large quantities) was to select based on SDSS-measured spectra \citep{Bernardi2003}. He knew, however, that this sort of method would exclude potential star-forming ellipticals (as well as potentially include passive spirals), due to the non-perfect correlation between morphology and color, as mentioned in the previous Chapter. So, realizing that a visual inspection of the direct appearance of the galaxies was necessary to create a complete sample of ellipticals independent of color, Schawinksi devoted an entire week to classifying 50,000 galaxies by eye (MOSES, \citet{Schawinski2007}). 

The grueling task of classifying only a small fraction of the entire SDSS main sample ($\sim$ 900,000 galaxies) made it apparent that a better method for visual classification was becoming neccessary. Inspired by the 20,000 volunteers who participated in the Stardust@Home project, Schawinksi and Oxford colleague Chris Lintott realized that it would only take a few years to classify all of SDSS Main, assuming a similar participation response as StarDust. This led to the launch of Galaxy Zoo in July, 2007. This first phase (known now as Galaxy Zoo 1, or GZ1), included the brightest (Petrosian magnitude $r < 17.77$ AB mag) 893,212 images from SDSS Data Release 6 \citep{Strauss2002,AdelmanMcCarthy2008}. In this project users were asked to indentify simple features of a given galaxy, including whether it was elliptical or spiral, clockwise or anticlockwise, a merger, or star/other (the original interface with options is shown in Figure~\ref{GZ1_Interface}). 

\begin{figure}
\centering
\includegraphics[width=.8\textwidth]{figures/GZ1Interface.png}
\caption{Example of the interface seen by users of Galaxy Zoo 1. On the left is an image of a galaxy from the SDSS main sample. On the right are possible features the user may identify about the galaxy by clicking the relevant option(s). Once complete, they are shown another galaxy.}
\label{GZ1_Interface}
\end{figure}

In just the first day of the site being live, 20,000 classifications were coming in each hour - a rate much faster than the developers had ever expected. In less than a year, the entire SDSS main sample was classified by an average of 38 volunteers per galaxy. Following GZ1 data release paper published in April, 2008 \citep{Lintott2008}, over a dozen scienific publications were released which made use of the morphological classifications \footnote{https://www.zooniverse.org/about/publications}. Significant results included the discovery of a large population of passive red spirals in the local universe \citep{Masters2010}, the existance of star-forming blue ellipticals \citep{Schawinski2009b}, ``green peas,'' a new class of galaxies exhibiting extremely high star formation \citep{Cardamone2009}, and ``Hanny's Voorwerp,'' the first example of AGN-photoionized clouds detected near galaxies no longer actively hosting AGN \citep{Lintott2009}.

It is also worth highlighting the educational impact of a citizen science approach to data collection. Science education research has shown that active participation is a critical component in scientific learning. \citet{Michaels2008} define four ``strands'' of skills that students must obtain to be considered scientifically proficient, the fourth being ``participating productively in science.'' Citizen science provides both students and the general public to actively participate in science without having to already be experts in the field, and it has been obvious so far that the volunteers are enthusiastic to do so. \citet{Raddick2010} investigated the motivations driving the participation of GZ users through surveys and interviews, and found the desire to contribute significantly to important research was one of the primary examples (other motivations including enjoying the beauty of the galaxy images and a general interest in astronomy). 


\begin{figure}
\centering
\includegraphics[width=0.95\textwidth]{figures/gz_interface_1.png}
\caption{Example of the interface seen by users of Galaxy Zoo 2. On the left is an image of a galaxy, on the right are possible features the user may identify about the galaxy by clicking the relevant option. Unlike GZ1, subsequent questions appear about the same galaxy depending on their answers to the preceding questions, following a decistion tree format (see Figure~\ref{fig:decisiontree} for a visual of all possible pathways.)}
\end{figure}

\begin{figure}
\centering
\includegraphics[width=0.95\textwidth]{figures/gzh_decision_tree.pdf}
\label{fig:decisiontree}
\caption{Decision tree used in the Galaxy Zoo:Hubble project. The colors indicate the ``Tier'' level of the question. Gray represents 1st-Tier; these are asked of all users. Green are 2nd-Tier; these are only asked after responding to a 1st-Tier question, and so on. This tree is identical to GZ2 and UKIDSS, except for the addition of the clumpy questions T12-T18.}
\end{figure}

The remainer of this Chapter will outline the common practices used to turn Galaxy Zoo data ``from clicks to classifications,'' through the use of consistency-weighting the user votes and adjusting vote fractions for redshift bias. 

\section{Galaxy Zoo Data Reduction}
\subsection{User weighting by consistency}
A typical Galaxy Zoo project collects classifications from over 10,000 unique volunteers. With such large numbers of classifiers, there exists the possibility that some fraction of these are ``unreliable'', that is, their votes are consistent with random clicking. To ensure that all votes collected represent real classifications, a weighting technique is implemented to detect and down-weight unreliable votes.

The weighting scheme used for all GZ projects represented in this thesis (GZ2, GZ:UKIDSS, and GZ:Hubble) evaluates the consistency of each user by how often their votes agree with the majority for each task in the decision tree. The consistency rating $\kappa$ for a single task is defined as:

\begin{equation}
\kappa = \frac{1}{N_{r}}\sum_{i=1}^{N_{r}}{\kappa_{i}}
\label{eqn:kappa}
\end{equation}

where $f_{r}$ is the vote fraction for each response in the task, $N_{r}$ represents the total number of responses to the task, $\kappa_{i} = f_{r}$ if the user's vote corresponds to response $i$, and $\kappa_{i} = (1-f_{r})$ if it does not. In this system, $\kappa$ is then high if the vote agrees with the majority, and low if it does not. 

The mean consistency computed for each response given is defined as the user's overall consistency $\bar{\kappa}$, and the user is assigned a weight $w$ defined as:

\begin{equation}
w = \rm min (1.0,(\bar{\kappa}/0.6)^{8.5})
\label{eqn:weight}
\end{equation}

All votes are then recalculated using the user weights, and the process is repeated as many as three times to ensure convergence. It can be seen in Equation~\ref{eqn:weight} that a user's weight value is always less than or equal to one; in other words, users are only downweighted in cases of noticeable inconsistency, and never upweighted. \citet{Willett2013} show that most users with low consistencies tend to only have contributed a handful of classifications, which could either indicate that users become more accurate as they classify more galaxies, or that inconsistent users are inherently less likely to be interested in the project. 

 
\subsection{Classification bias in the local Universe}

For samples of galaxies limited to the local universe ($z\lesssim0.2)$, there is no expected redshift dependence on the morphological classifications. Therefore, we would expect vote fractions representing different morphological features to be constant with respect to redshift. However, this is not the case - the average vote fraction for features, bars, spirals, and several others actually tend to \emph{decrease} with redshift. Since we assume such features should be equally prevalent at any redshift in this small range, some bias unrelated to any true morphological evolution must be affecting the vote fractions. 

The source of this bias comes from the apparent size and brightness of the images of the galaxies being classified, which are strongly affected by redshift. Images of more distant galaxies appear smaller and dimmer, and therefore finer features are simply more difficult to detect. This sort of classification bias is a problem with any morphological classification, whether it be expert classifiers, automated dection, or crowd-sourced visual inpsection.

This section will describe the methods used to correct this type of classification bias for galaxies in the local Universe, where no true morphological evolution is a factor. Beyond the local Universe this assumption is no longer valed, so techniques implimenting classifications of artificially-redshifted galaxies are used for calibration; these are described in detail in Chapter~\ref{chap:ferengi}. 

\subsubsection{Debiasing Galaxy Zoo 2: W13 method}

The first technique for debiasing Galaxy Zoo classifications was developed by \citet{Bamford2009}. This method was used again for the GZ2 classifications, with slight modifications to account for 1) the GZ2 classifications were derived from votes through a decision tree, rather than a single response per galaxy, and 2) answers to tasks in GZ2 are not all binary as they were with GZ1. This section will describe the technique in the context of GZ2, noting that the physical assumptions used are the same in both methods.  

The debiasing technique used in GZ2 assumed firstly that galaxies with similar brightnesses and sizes should, on average, share similar mixes of morphologies at any redshift. Using this assumption, galaxies were grouped into bins of absolute magnitude $M_r$, Petrosian effective radius $R_{50}$, and redshift. For each task in the GZ2 decision tree, the vote fractions for each response in any size/magnitude bin were adjusted so that their average matched the average vote fraction of its lowest-redshift bin. This method is described in detail in \citet{Willett2013}, but the main approach is as follows:

For a given size/magnitude bin, the ratio of vote fractions for a pair of responses $i$ and $j$ for a single task can be written as $f_i/f_j$. Due to the classification bias described above, this ratio may not reflect the ``true'' ratio for this size/magnitude range, but can be written in terms of the true ratio with a multiplicitave constant $K_{i,j}$:

\begin{equation}
\left(\frac{f_i}{f_j}\right)_{z=z'} = \left(\frac{f_i}{f_j}\right)_{z=0} \times K_{i,j}
\label{eqn:fvspk}
\end{equation}

Where $(f_i/f_j)_{z=z'}$ represents the ratio measured in a size/magnitude bin at $z=z'$, and $(f_i/f_j)_{z=0}$ is the ``true,'' or intrinsic ratio of vote fractions, defined as the ratio measured in the lowest redshift bin.

\begin{figure}
\centering
\includegraphics[width=\textwidth]{figures/gz2debiased.pdf}
\caption{Local ratios of morphologies for the first three tasks in the GZ2 decision tree, used to derive debiased votes for the GZ2 sample. The full figure which includes baseline ratios for all tasks in the GZ2 decision tree is shown in \citet{Willett2013}, Figure 5.}
\label{fig:gz2debiased}
\end{figure} 

Figure~\ref{fig:gz2debiased} shows the local ($z=0$) ratios of $f_i/f_j$ for the first two responses $i$ and $j$ for the first three tasks of the GZ2 decision tree, which are used to calculate the debiased vote fractions as outlined above. For Task 01, $f_i/f_j$ corresponds to $f_{smooth}/f_{features}$, for Task 02 $f_{edgeon}/f_{not~edgeon}$, and for Task 03 $f_{bar}/f_{no~bar}.$ The figure demonstrates the size and magnitude dependence of the most local morphological populations: for example, in Task 01, the largest and brightest galaxies tend to have more votes for ``smooth'' than ``featured'', which is consistent with our current understanding that ellipticals tend to be larger and more massive than spirals. 


\begin{figure}
\centering
\includegraphics[width=\textwidth,trim={0cm 2.7cm 0cm 0cm},clip]{figures/GZvotefractions.pdf}
\caption{Credit: \citet{Hart2016}, Figures 8 and 9. \textbf{Top}: Plotted are the fraction of galaxies with vote fractions greater than 0.5 for each reponse to the first 3 tasks, where the solid lines are the raw vote fractions, dotted are the W13 debiased vote fractions, and dashed-dotted lines are the debiased with the H16 method. As an example of the effect of the debiasing, see panel (a): without the debiasing, the number of galaxies with a ``smooth'' majority vote fraction increashes sharply from $z=0.04$ to $z=0.8$, a range assumed to be local enough such that no true morphological evolution should be observed. Both debiasing methods work to keep the fractions constant over this redshift range, although the H16 method is more effective at higher-tier questions. \textbf{Bottom}: Distributions of vote fractions for the first answer to the first 3 tasks, for the low-redshift raw data (solid blue), high redshift raw data (black solid line), W13 debiased (red thin-dashed line), and H16 debiased (red thick-dashed line). Both methods are successful at shifting the high-redshift distributions to match the low-redshift distribution, with H16 being slightly more effective at matching the shape of the distributions.}
\label{fig:gz2debiasingresults}
\end{figure}

%this paragraph is shit fix it
The results of this method for the first three Tasks in the GZ2 decision tree are shown in Figure~\ref{fig:gz2debiasingresults}. For each response in each Task, the average vote fraction is calculated as a function of redshift. Solid lines represent the weighted/non-debiased votes and the dotted lines are the debiased votes using this method (hereafter W13).
The redshift dependence on vote fraction is very evident in the downward trend of the solid lines corresponding to responses which detect features, such as $f_{features}$ and $f_{bar}$ in this example. The dashed lines show the effect of the debiasing which attempts to flatten out the distribution. Full figures showing the results for all Tasks in the tree are available in \citet{Willett2013} (Figure 3) and \citet{Hart2016} (Figure 8). From 2013-2017, the debiased vote fractions calculated in this method were used in the majority of published Galaxy Zoo papers, and are used in the study described by Chapter~\ref{chap:baragn}. 


\subsubsection{Debiasing GZ2 and UKIDSS: H16 method}

The W13 debiasing method is successful at adjusting the vote fractions to more accurately resemble the ``true'' distribution of morphologies at low redshift, but has two primary limitations. First, the rectangular binning of all three parameters (size, magnitude, and redshift) is only effective when the parent sample is large enough that sufficient data per bin remains available after the three dimentional binning. (For example, to requre 10 bins in each parameter with at least 50 galaxies per bin, a parent sample must contain at minimum N=10x10x10x50=50,000 galaxies, assuming a perfectly even distribution of values in each parameter). GZ2 is not so affected by this limitation, with a parent sample size of $\sim$ 250,000 galaxies. However, this is only true when considering the debiasing of the first Task, which is asked of every galaxy. After this Task, the parent sample for computing a correction term decreases as not all Tasks are asked of every galaxy; for example, the Tier 4 Task which asks for the number of spiral arms is only seen by the majority of volunteers in 33,000 galaxies of the full GZ2 sample. Thus debiasing this Task would require a smaller limit on the number of bins per dimension or the number of galaxies per bin, both of which decrease the robustness of the method. Even with a large parent sample for any Task, the rectangular binning is also limited by the inability to account for data which lie on the outer edges of the parameter space, as there tends to be insufficient data in the outer bins. 

A new debiasing technique (hereafter H16) was developed by Galaxy Zoo member Ross Hart \citep{Hart2016} which substitutes Voronoi binning for the rectangular method. Voronoi binning optimizes the shape and location of bins based on the desired signal for each bin; in this case, the number of galaxies per bin is set initially, and the bins are drawn to fulfill that requirement. In this way, the number of galaxies available for measuring the change in vote fractions for each bin is maximized. Thus, this method is more effective at debiasing smaller samples (such as GZ:UKIDSS which contains only $\sim$70,000 galaxies; see Chapter~\ref{chap:ukidss}), where the three dimensional binning preserves the signal in each bin. An example of Voronoi binning GZ2 data in size and magnitude is shown in the left panel of Figure~\ref{fig:voronoi}. Each size and magnitude bin is then Voronoi-binned by redshift. 

The second limitation of the W13 method is that while it effectively corrects the vote fractions for any Task so that the average morphology is constant as a function of redshift, it does not account for the \emph{distribution} of morphologies at low redshift. This produces good results when the corrected values are used for population studies, where the percentage of galaxies exhibiting a particular morphology are desired, but may not always reproduce accurate \emph{individual} vote fractions. The R16 method instead corrects the high redshift vote fractions based on the change in distribution of vote fractions observed at low redshift, rather than comparing to only the average values. The first step of this method is shown in the right panel of Figure~\ref{fig:voronoi}. For the low redshift bin of a given task, the cumulative distribution of vote fractions for each response is fit with a continuous function, which is used as the baseline distribution (similar to teh baseline average votes in the W13 method.) The vote fractions making up the cumulative distributions at higher redshifts are then adjusted as needed to match the low redshift distribution as closely as possible.


\begin{figure}
\centering
\includegraphics[width=\textwidth,trim={.5cm .25cm .5cm .5cm},clip]{figures/Voronoi.pdf}
\caption{Credit: \citet{Hart2016}, Figures 5 and 6. \textbf{Left:} Voronoi bin distribution for the ``$>4$'' answer to the spiral arm question in GZ2. Each bin is further divided into Voronoi bins, such that each final $R_{50}-M_{r}-z$ bin contains at least 50 galaxies. \textbf{Right:} Cumulative distribution of vote fractions (in log-space) of a single $R_{50}-M_{r}$ bin, split between a high redshift bin (red dashed line) and a low redshift bin (blue solid line). The debiasing method adjusts the high-redshift vote fractions to match the distribution of the low-redshift distribution. }
\label{fig:voronoi}
\end{figure}

Results of this method are shown and compared to W13 in Figure~\ref{fig:gz2debiasingresults}. Plotted on the top panel are the fraction of galaxies with vote fractions greater than 0.5 for each reponse to the first three tasks, where the solid lines are the raw vote fractions, dotted are the W13 debiased vote fractions, and dashed-dotted lines are the debiased with the H16 method.  Both methods are succesful in stabilizing the average morphologies over this local redshift range. The bottom panel shows the distribution of vote fractions of a low-redshift bin (solid blue histogram) and high redshift bin, again for the raw votes (black solid line), W13 method (light dashed red), and H16 method (solid dashed red). It can be seen here that while both methods can reproduce the average vote fractions at low redshift, the H16 method is more succesful in reproducing the distribution of votes at low redshift. In this thesis, W13 debiased vote fractions were used in Chapter~\ref{chap:baragn} to conduct the study examining the relationship between bars and AGN using GZ2 data, specifically votes from the first three Tasks in the tree. Chapter~\ref{chap:ukidss} examines a smaller data set, the UKIDSS sample, which contains 70,000 galaxies, much smaller than GZ2. For the reasons given in this section, the H16 method was used to debiase the votes used in that study.

The science in Chapter~\ref{chap:gzh_red_disks} examines galaxies residing far beyond the local Universe ($0.2\le z \le 1.0$), whose morphologies were classified classified in the Galaxy Zoo: Hubble project, using images from the HST-Legacy surveys. One of the key assumptions in the local-Universe debiasing techniques outlined in this chapter was that no significant morphological evolution exists in that redshift range. This is not a valid assumption for high-redshift galaxies, which are known to exhibit very different morphological populations at earlier epochs. A new debiasing technique was thus developed for the GZH data catalogue, using classifications from artificially-redshifted galaxies to quantify the effect of the redshift bias. The next Chapter will describe the creation and implementation of the simulated data set into the debiasing method applicable for high-redshift galaxies. 
% end of chapter






